
\documentclass[a4paper,11pt]{article}

\usepackage{physics}

\usepackage{listings}
\lstset{
    basicstyle=\ttfamily
}
\def\co{\lstinline}

\usepackage{biblatex}
\addbibresource{qs.bib}

\title{\(\mu\), A Minimal Patch for Quantum Computation, Information and Foundation}
\author{Sajad}
\date{\today}

\begin{document}

\maketitle


\begin{abstract}
Today, as quantum computing is trending more and more, while companies and universities are making different tools for different purposes, the lack of a general-purpose, well-designed, and futuristic programming language is felt.
Here is a try to design a patch to be added to (almost) any programming language to make it support quantum algorithms and simulations. Moreover, this definition may help to represent clearer not only quantum algorithms but also many other problems in quantum.
Although the design process was inspired by theoretical computer science concepts, specifically PL concepts, it tries to be intuitive and friendly for mathematicians and physicists.
The process of design, presented specifications and details of implementation is discussed in this paper. While the real examples are available at Github.
\end{abstract}
\clearpage

\tableofcontents
\clearpage

\section{Introduction}

This text, assumes you have proper knowledge in quantum computing. If you don't, you can read \cite{neilson} or any other standard books.

\section{Purpose}
\begin{itemize}
\item Easier way to write, read and understand QC algorithms

The way we write down quantum algorithms is seemed to be not a projection of what we understand from them. Often learning every single quantum algorithm need to be a task. That's maybe not just for its intrinsic hardness, and the way we often propose algorithms (mathematics for the definition of gates, and a circuit for its sequence) is disjointed and hard to follow. Maybe a more concrete way to represent makes them easier to understand, modify, and generalize.

For that reason, we need to just an standard way to write down a psuedo that uniquely specifies the algorithm. But other purposes give us the reason why we need to implement such thing.

\item Intuitive definition and computation for scientists

People with mathematical backgrounds (mathematicians and physicists) are often having hard time learning a simulation/expermental tool, specially when it's designed by developers. this came from the different intuitions that these two groups have.
A simulation/experimental tool must be at the same abstraction level that scientists are familiar with to be a comfortable. Another point is that, some may belive that modules and organizing functionalities in group are not compatible with scientific idea of solving problem, it may be correct, but it's clear that ability of importing and having all tools together, is not necessarily logically in contract with organizing them as groups at the first place.

It's not necessary to point that using the scientific terms, operators, and functions are the baseline to do to make it scientist-friendly.

\item a Swiss army knife, not a giant supertool
% TODO

\textbf{NOT COMPLETED}
A minimal and complete set of tools,

Fully understandable for developers, engineers and scientists.

\item Layered and extensible patch

\item futuristic
\end{itemize}

\section{Design}

\subsection{Basic Hilbert Space Interface (BHSF)}

Before gettin into the main subject, we offer a library with a common set of actions on
Hilbert space.
This is like the effort people made before to design BLAS (Basic linear algebra system),
defining fundamental tools that are necessary for those kind of computations.

Structurally, BHSF is guessed defined be over a BLAS. Here we just define the interface
in term of types, functions and a brief structure (which is technically implmentation, not definition)


\textbf{Types: }
\begin{itemize}
    \item \co|H| is the type that represents a single Hilbert space (tensor-product spaces are not counted in).
	\item \co|Vector| which represents a vector in a set/list of Hilbert spaces.
\end{itemize}

\textbf{Functions: }

\begin{itemize}
	\item \co|dual(H)| \(\to\) \co|H|:
	    This function, returns dual of the space. This is useful because an operator on \co|H| is like a vector in the \co|dual(H)| \(\otimes\) \co|H|.

	    So we ask this function to satisfy \co|dual(H)| \(\ne\) \co|H| and \co|dual(dual(H))| \(\ne\) \co|H|.%
	    \footnote{We know that mathematically \(H^{**}\) double dual of a space is not equal but isomorphic to itself.
            This fact is useful in the implementaton of a general library to support channels and superoperators with a simple dirac-like notation.}
    \item \co|idual(H)| \(\to\) \co|H|:
        inverse of the \co|dual|.
    \item \co|dim(H)| \(\to\) \co|Int|:
        Returns dimension of the \co|H|.
    \item \co|space(Vector)| \(\to\) \co|List[H]|:
        This function returns list of Hilbert spaces that this vector belongs to tensor product of them.
        This is just used to discuss other functions.%
        \footnote{Although we ignore the isomorphism between \(H^{**}\) and \(H\), we use this fact that $H_A \otimes H_B$ is ismorphic to $H_B \otimes H_A$ and we assume the isomporphism is identity. therefore \co|space| function will not guarantee the order.}
	\item \co|dual(Vector)| \(\to\) \co|Vector|:
	    This function returns dual of the vector.
	\item \co|idual(Vector)| \(\to\) \co|Vector|:
	    This function returns inverse dual of the vector.
	\item \co|pnorm(Vector, p: Real)| \(\to\) \co|Real|:
	    This function returns $\ell^p$-norm of the vector.
	\item \co|tr(Vector, List[H])| \(\to\) \co|Vector|:
	    Trace of the vector over a list of Hilbert spaces.
\end{itemize}

\textbf{Operators: }

\begin{itemize}
    \item \co|Vector * Vector| \(\to\) \co|Vector|:
	    Tensor contraction similar to dirac notation. It means that for \co|a, b: Vector|,
	    If \co|space(a)| = \co|[dual(H)]| and \co|space(b)| = \co|[H]| then \co|space(a * b)| = \co|[]| and \co|space(b * a)| = \co|[dual(H), H]|. And obviously if \co|space(a)| = \co|[|\(\mathcal{H}_1\)\co|]|  and \co|space(b)| = \co|[|\(\mathcal{H}_2\)\co|]|
	    then \co|space(a * b)| = \co|space(a * b)| = \co|[|\(\mathcal{H}_1\)\co|, |\(\mathcal{H}_1\)\co|]|.
	\item \co|Vector - Vector| \(\to\) \co|Vector|: subtraction of two vector.
    \item \co|Vector * Complex| \(\to\) \co|Vector|
    \item \co|Complex * Vector| \(\to\) \co|Vector|
    \item \co|Vector / Complex| \(\to\) \co|Vector|: scalar multiplication and division.

\end{itemize}

\textbf{Conversions: }

\begin{itemize}
	\item \co|Vector| \(\leftrightarrow\) \co|Complex|: vectors \co|v| with \co|space(v)| = \co|[]|, are isomorphic to complex numbers.
\end{itemize}

\textbf{Constructors: }

\begin{itemize}
	\item \co|H(dim: Int)| \(\to\) \co|H|: For constructing \co|H| from its dimension.
	\item \co|Vector(v: List[Complex], H)| \(\to\) \co|Vector|: This function constructs a vector from a list of coefficents each corresponds to each basis of \co|H|, it's like $\sum_i v_i \ket{i}$
\end{itemize}

\textbf{Example}

The following psuedo-code shows the functionality of BHSF.

\begin{lstlisting}
h = H(3)
v1 = Vector([1,0,0], h)
v2 = Vector([0,1,0], h)

dual(v1) * v2) ; same as <v1|v2>
v1 * dual(v2)  ; same as |v1> <v2|
\end{lstlisting}

\subsection{Core}

By adding a special generic type called $Q$, to any modern programming language we can make it possible.

\co{Q<T>} (C/C++/Java) \co{Q[T]} (Scala) \co|Q{T}| (Julia) means a quantum space in $\mathcal{H}(T)$ calling Hilbert space with $T$ as basis. But we know that having multiple \co{Q}-typed values, makes our Hilbert space much larger that is kronoker product (tensor product) of spaces/vectors.

\subsubsection{\co|Q|-type}
While physicists/mathematicians think about groups, maps and measures, within different levels of abstraction, and computer scientists think about primitives, structs, datastructures and more, the major portion of quantum computing tools are offering "Register Arrays" as the main tool for computation, which is disappointing for anyone other digital systems experts.

\subsubsection{Views}
Assuming \co|a: Q{T}|, \co|b: Q{V}| a type, \co|(a, b): Tuple{Q{T}, Q{V}}| can be intrepreted as \co|Q{Tuple{T, V}}|, this is called viewing, or more specifically, cross view.

Also, for a compound (cartesian-cross) type, such as static-length arrays, tuples, structs, a projection can have a similiar viewing called projection view.

A more quantum view, is to use another basis (the same or of another type, but with the same cardinality) as another representation for a vector, an extreme example is spins in different axes or total spin basis.

\subsubsection{Multiview Type}

When we have a quantum hilbert space that we can assign different basis to it, while we shall not promote one of them as the main, we can use union, an old abondened tool in programming. But if you're not familiar with it, we just use a compound structure, assigning different basis views to its fields.

\subsubsection{Mutability}

This are not mutable, but this is not bad, non-copying principle is not compatible with copying in closure and more in immutable functional programming.

but in another hand we can present another meaning for mutability / immutability,
views can be read-only or read-write, those we can set them mutable / immutable.

\subsubsection{Move Semantic}

\subsubsection{Operators and Superoperators}
Just by using match-case and some automatically applied castings, it can be intuitively written.

\subsection{Code as Operators}
As a beloved target, codes in the operators that are enslaved within lambda expression, may can be freely written.

\section{Implementation}
\subsection{Julia}
\section{Examples}
\subsection{Algorithms}
\subsubsection{BV}

\begin{lstlisting}

def BV(func, n)
    inputs = Q((0, ) * n)
    result = Q(false)
    
    for i in inputs:
        i = i.apply(u(lambda x: ket(0) + (-1) ** x * ket(1)))
    
    result = t(result, inputs).apply(u(lambda (x, y) : y + func(x))
    
    return result.measure() 

gate<n> fourier = match n
    case 0:
    otherwise: fourier<n - 1>

optional<T> grover(f: pure<bool, T>, init: set<T>, real p) {
    q<T> vec = init;

    gate reflect = x: T => match x
        f(x) ==1, 1
                  -1

    vec -> (reflect f)^(1/p)
    T res = vec -> measure
    if f(res)
        return Some(res)
    else
        return None
}

real QAOA<T> problem(f: pure<bit[], real>, int level) {
    q<T> vec;
    hamiltonian C = t : T => f(t)
    hamiltonian B = X[n]
    vec -> exp(iC) -> exp(iB)
}

real annealing(f: problem,

\end{lstlisting}



\end{document}
